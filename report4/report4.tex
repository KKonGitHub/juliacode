\documentclass{ltjsarticle}

\usepackage{graphicx}
\usepackage{physics}
\usepackage{amsmath}
\usepackage{here}



\begin{document}

\Huge
    \begin{center}
        数値計算法第四回授業レポート

        04B21024
        葛堀和也
    \end{center}
\clearpage

\normalsize

\section{問1}
次の方程式をNLsolveを使った方法と、二分法の両方で解く問題であった。
\begin{align}
    \sin(x) = \frac{3}{4}
\end{align}

まず二分法によって近似解を得た。以下の表は二分法による解法の初期値や試行回数上限などの値である。
\begin{table}[H]
    \centering
    \begin{tabular}{c|cc}
        \hline \hline 
        試行回数上限 & MAX & 100\\
        許容誤差 & err & 1e-8\\
        初期値上限 & top & \text($\frac{\pi}{2}$)\\
        初期値下限 & bottom & 0\\
        \hline \hline        
    \end{tabular}
\end{table}

それによる結果は次のようであった。
\begin{center}
近似解:0.8480620808468514
\end{center}
また、この近似解に至るまでのステップ数は27回であった。

次に、NLsolveを用いた解法によって近似解を求めた。初期値は1.0とした。
この時、近似解は収束して、
\begin{center}
    近似解:0.848062078981481
\end{center}
となった。
この2つを比較すると、9桁まで一致していることがわかる。
二分法においては誤差を8桁以下としていたので、これとほぼ一致する。
したがって、8桁まで信頼できる解である。


\section{問2}

次の非線形方程式を解く。
\begin{align}
    \begin{bmatrix}
        x + 2y + z\\
        x^{2} + y^{2} + z^{2}\\
        \sin(x+y+z)
    \end{bmatrix}
    = 
    \begin{bmatrix}
        -1\\
        10\\
        0.7\\
    \end{bmatrix}
\end{align}

この式を変形して、次のようにする。
\begin{align}
    F = 
    \begin{bmatrix}
        x+2y+z+1\\
        x^{2} + 2y^{2} + z^{2} -10\\
        \sin(x+y+z)-0.7
    \end{bmatrix}
    =
    \begin{bmatrix}
        0\\
        0\\
        0
    \end{bmatrix}
\end{align}

左辺を、その解を求めるべき関数$F$とみなす。このベクトル値関数のヤコビ行列は以下の通り。

\begin{align}
    \nabla F = 
    \begin{bmatrix}
        1 & 2 & 1 \\
        2x & 4y & 2z \\
        \cos(x+y+z) & \cos(x+y+z) & \cos(x+y+z)
    \end{bmatrix}
\end{align}
さて、これらの式より、ニュートン法を用いてこの非線形方程式を近似的に解くことができる。

初期値や指定した値は以下の通り。
\begin{table}[H]
    \centering
    \begin{tabular}{c|cc}
        \hline \hline
        試行回数上限 & MAX & 100\\
        誤差上限 & err & 1e-7\\
        初期値 & temp & [1.0 ,0.0 ,0.0]\\
        \hline \hline
    \end{tabular}
\end{table}

これによる近似解は以下のようであった。
\begin{center}
    近似解:[1.7458493588323, -1.775397496610753, 0.804945634389206]
\end{center}
ただし、左から順に、x,y,z成分の解を表す。

次に、これをNLsolveパッケージの関数を用いて近似的に解くことを考える。
%まだできてない。例のエラーが出る。






\end{document}