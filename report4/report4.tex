\documentclass{ltjsarticle}

\usepackage{graphicx}
\usepackage{physics}
\usepackage{here}



\begin{document}

\Huge
    \begin{center}
    数値計算法第五回授業レポート

    04B21024
    葛堀和也
    \end{center}
\clearpage

\normalsize

\section{問1}
$\Delta t$の満たす必要のある条件として、以下がわかっている。

\begin{align}
    \Delta t < \frac{\pi - u_{n}}{\gamma (\sin(u_{n}))^{1.2}}
    \label{eq:base}
\end{align}
$u_{n}$は自然数であるので、全ての自然数に対してこの条件を満たすような$\Delta t$を選べば良い。
したがって、次の関数の最小値を求める必要がある。

\begin{align}
    Up(u) = \frac{\pi - u}{\gamma (\sin(u))^{1.2}}
    \label{eq:cfunc}
\end{align}


この関数について考察する。まず、この関数は原点で正の無限大に発散することがわかる。
というのも、$\sin(u)$が$u \rightarrow 0$で0になるからである。
全く同様の理由で、そこから$\sin$の半周期分だけずれた点$u=\pi$でもこの関数は正の無限大に発散するとわかる。
というのも、
\begin{align}
    Up(u) &= \frac{\pi - u}{\gamma (\sin(u))^{1.2}}
          &= \frac{\pi - u}{\gamma \sin(u)}  \times \frac{1}{(\sin(u))^{0.2}}
          &= \frac{\pi - u}{\gamma \sin(\pi -u)} \times \frac{1}{\sin(u)^{0.2}}
\end{align}
これより、前半部分は$u \rightarrow \pi$の極限をとると1になるが、一方で後半部分はそのまま正の無限大に発散する。

関数\ref{eq:cfunc}には他に極地はなく、しかもこの範囲、すなわち$0<u<\pi$の範囲で連続かつ微分可能であるので、
この範囲における極小値が、この範囲における最小値となる。

\subsection{手計算}
最初に手でやってみる。
\begin{align}
    \frac{d}{du}Uq = 
\end{align}






\end{document}